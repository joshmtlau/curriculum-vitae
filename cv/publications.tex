%-------------------------------------------------------------------------------
%	SECTION TITLE
%-------------------------------------------------------------------------------
\cvsection{Publications}


%-------------------------------------------------------------------------------
%	CONTENT
%-------------------------------------------------------------------------------
\begin{cventries}

%---------------------------------------------------------
    \cventry
        {Mu-Te Lau, Hsiang-Chun Yang, Hsin-Yu Chen, Chung-Yang (Ric) Huang}
        {A Lazy Resynthesis Approach for Simultaneous T Gate and Two-Qubit Gate Optimization of Quantum Circuits
        $|$ 
        \href{https://arxiv.org/abs/2508.04092}{arXiv \faLink}}
        {National Taiwan University, Taiwan}
        {Sep. 2025, To appear on IEEE QCE 2025}
        {
            \begin{cvitems}
               \item \textcolor{awesome}{\textbf{Reduced 2Q-count overhead by 54.8\% for tableau-based quantum circuit optimization while achieving 1.81$\times$ speedup}}
               \item A more scalable approach to ZX-calculus-based optimizations while yielding comparable 2Q-counts
            \end{cvitems}
        }
        
    \cventry
        {Mu-Te Lau (Advisor: Chung-Yang (Ric) Huang)}
        {Multi-Objective Quantum Circuit Optimization by Combining Tableau-Based and ZX-Diagram-Based Techniques
        $|$ 
        \href{https://tdr.lib.ntu.edu.tw/handle/123456789/95208}{Master's Thesis \faLink}}
        {National Taiwan University, Taiwan}
        {Jul. 2024, Master's Thesis}
        {
            \begin{cvitems}
               \item \textcolor{awesome}{\textbf{Proposed a hybrid QCO flow for Clifford+T circuits that give a 29.4\% improvement in 2Q-counts over purely tableau-based flows}}
               \item Revealed a trade-off between the choice of data structures that influence the optimization of two-qubit gate counts and T/H- gate counts
            \end{cvitems}
        }
        
    \cventry
        {Mu-Te Lau, Chin-Yi Cheng, Cheng-Hua Lu, Chung-Yang (Ric) Huang (Corresponding Author), et al.}
        {Qsyn: A Developer-Friendly Quantum Circuit Synthesis Framework for NISQ Era and Beyond 
        $|$ 
        \href{https://arxiv.org/abs/2405.07197}{arXiv \faLink} 
        $|$ 
        \href{https://github.com/DVLab-NTU/qsyn}{\faGithub 160+ \raisebox{0.1em}{\textbf{$\star$}}}}
        {National Taiwan University, Taiwan}
        {Apr. 2024, Preprint}
        {
            \begin{cvitems}
               \item \textcolor{awesome}{\textbf{Poster presented on \href{https://qce.quantum.ieee.org/2024/posters-program/}{IEEE QCE 2024 in Montr\'eal, Canada} and \href{https://quantum-compilers.github.io/iwqc2024/posters.html}{6th IWQC in Berlin, Germany}}}
               \item A fast, modular, and research-backed open-source framework for quantum circuit synthesis
            \end{cvitems}
        }
        
\end{cventries}