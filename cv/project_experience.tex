%-------------------------------------------------------------------------------
%	SECTION TITLE
%-------------------------------------------------------------------------------
\cvsection{Project Experiences}


%-------------------------------------------------------------------------------
%	CONTENT
%-------------------------------------------------------------------------------
\begin{cventries}
    \cventry
        {Quantum Computing; Modern C++; Docker} % Job title
        {Qsyn 
        $|$ 
        \href{https://arxiv.org/abs/2405.07197}{arXiv \faLink} 
        $|$ 
        \href{https://github.com/DVLab-NTU/qsyn}{\faGithub} 160+ \raisebox{0.1em}{\textbf{$\star$}}} % Organization
        {National Taiwan University, Taiwan} % Location
        {2022 Fall--Now} % Date(s)
        {
          \begin{cvitems} % Description(s) of tasks/responsibilities
            \item \textcolor{awesome}{\textbf{Reimplemented and improved QCO algorithms to assess for scalable, high-performance quantum circuit synthesis}}
            \item Implemented a flexible command-line interface to combine QCO algorithms flexibly
            \item Coordinated refactorings to core data structures to ensure code quality and flexibility
            \item Guided new team members with their contributions and taught them good coding practices
          \end{cvitems}
        }

    \cventry
        {JS/React; MongoDB; Docker} % Job title
        {\textbf{Design Verification Lab Website $|$ \href{https://dvlab.ee.ntu.edu.tw/}{\faLink}}} % Organization
        {National Taiwan University, Taiwan} % Location
        {2021 Spring} % Date(s)
        {
          \begin{cvitems} % Description(s) of tasks/responsibilities
            \item Developed a new website with other labmates
            \item Enhanced web development skills, esp. in implementing data flow
          \end{cvitems}
        }

    
    % \cventry
    %     {An almost-functional complete, minimum-viable implementation of the quantum-aware k-LUT mapping $|$ \emph{Python}} % Job title
    %     {\textbf{Q-Aware $k$-LUT} $|$ \href{https://github.com/DVLab-NTU/112-1-qda-hw3-q-aware-k-lut}{\faGithub}} % Organization
    %     {National Taiwan University, Taiwan} % Location
    %     {2023 Fall} % Date(s)
    %     {
    %       \begin{cvitems} % Description(s) of tasks/responsibilities
    %         \item Prototyped a logic network partitioning algorithms targeting the synthesis of quantum Boolean oracles
    %         \item Course material to the \emph{Special Topics on Quantum Design Automation} course that I TA'ed
    %       \end{cvitems}
    %     }

    \cventry
        {\emph{Survey, C++}} % Job title
        {\textbf{ZX-Diagrams as Intermediate Representation for Lattice Surgery Compilation}} % Organization
        {National Taiwan University, Taiwan} % Location
        {2022 Spring--2023 Summer} % Date(s)
        {
          \begin{cvitems} % Description(s) of tasks/responsibilities
            \item Term projects of the courses \emph{Fault-Tolerant Computing} and \emph{Quantum Information and Computation}
            \item \textcolor{awesome}{\textbf{Selected to be Exemplar Presentation Videos in the \emph{2022 Quantum Information and Computation} Course}}
            \item Compiled Fault-Tolerant Quantum Circuit to Lattice Surgery with ZX-calculus-based methods
            \item Achieved compact compilation results for quantum circuits with a small number of qubits
          \end{cvitems}
        }
\end{cventries}